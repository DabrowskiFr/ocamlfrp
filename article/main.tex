\documentclass[acmsmall,screen,review,nonacm]{acmart} %anonymous
\usepackage{listings}
%\usepackage{amsmath}
%\usepackage{amssymb}
\usepackage{xspace}
\newcommand{\letin}[3]{{\tt{let}}~#1=#2~{\tt{in}}~#3}
\newcommand{\YArrow}{\texttt{arr}\xspace}
\newcommand{\YFirst}{\texttt{ first}\xspace}
\newcommand{\YSecond}{\texttt{second}\xspace}
% \newcommand{\YCompose}{\texttt{>\!>\!>}\xspace}
\newcommand{\YAnd}{\texttt{\&\!\&\!\&}\xspace}
\newcommand{\YLoop}{\texttt{ loop}\xspace}
\newcommand{\YSF}[2]{{\tt{sf}}~#1~#2}
\newcommand{\stream}{{\tt{stream}}}
\newcommand{\YCompose}{\mathbin{\texttt{>\!>\!>}}}

\newcommand{\co}[2]{{\tt{co}}~#1~#2}
\newcommand{\raw}[2]{{\tt{raw}}~#1~#2}
\newcommand{\Co}[2]{{\tt{Co}}~#1~#2}

\def\unitval{{\tt{tt}}}
\def\unittype{{\tt{unit}}}

\setcopyright{acmlicensed}
\copyrightyear{2018}
\acmYear{2018}
\acmDOI{XXXXXXX.XXXXXXX}


\title{Arrowized Functional Reactive Programming in OCaml}
%\authornote{Both authors contributed equally to this research.}
\email{frederic.dabrowski@univ-orleans.fr}
\orcid{1234-5678-9012}
\author{Frédéric Dabrowski}
\authornotemark[1]
\affiliation{%
  \institution{Université d’Orléans, INSA CVL, LIFO, UR 4022}
  \city{Orléans}
  \country{France}
}

\begin{document}
\maketitle
Yampa is a Haskell library for programming reactive systems. It is based
on the concept of stream functions which takes an input stream and produce an output stream.
Streams are infinite sequences of values and stream functions are non terminating functions.
The lazyness of Haskell allows to implement streams and stream functions regardless of their 
infinite nature. 
In a strict language like OCaml this is not possible instead of using lazyness extensions.
Here we propose an implementation of Yampa primitives in OCaml based on coiterators.

\section{Yampa}
AFRP is based on John Hughes's Arrow framework, where signals are treated as second-class values and can only be accessed through signal functions.
Given two types \(a\) and \(b\), the type of signal functions \(\YSF{a}{b}\)
is defined as $\YSF{a}{b} = a \rightarrow b \times \YSF{a}{b}$.
A signal function of this ty    pe consumes the head of a stream of type \(a\),
produces a value of type \(b\), and returns a new signal to handle the remainder of 
the stream. The combinators listed below can be used to construct signal functions in AFRP.
\begin{equation*}
    \begin{array}{lcl}
        \YArrow & : & (a \rightarrow b) \rightarrow \YSF{a}{b}\\
        \YArrow\ f & = & \lambda a. (f\ a, \YArrow\ f)\\\\
        \YFirst & : & \YSF{a}{b} \rightarrow \YSF{(a \times )}{(b\times c)}\\
        \YFirst\ F & = & \lambda (a,c). \letin{(b,F')}{F a}{((a,c), F')}\\\\ 
        (\YCompose) &:& \YSF{a}{b} \rightarrow \YSF{b}{c} \rightarrow \YSF{a}{c}\\
        F \YCompose G & = & \lambda a. \letin{(b,F')}{F\ a}{\letin{(c,G')}{G\ b}{(c, F' \YCompose G' )}}\\\\
        \YLoop & : & \YSF{(a \times c)}{(b \times c)} \rightarrow c \rightarrow \YSF{a}{b}\\
        \YLoop\ F\ v & = &  \lambda a. \letin{((b,v'), F')}{F\ (a,v)}{\YLoop\ F'\ v'}
    \end{array}
\end{equation*}
Note that, by construction, the new stream function differs from the original only in the parameters of the 
\(\YLoop\) combinators. 
The function \({\tt{run}}\), shown below, converts a stream function 
\(\YSF{a}{b}\) into a function that operates on streams.
\begin{equation}
    \begin{array}{lcl}
        {\tt{run}} & : & \YSF{a}{b} \rightarrow \stream\ a \rightarrow \stream\ b\\
        {\tt{run}}\ F\ (a \cdot s) & = & \letin{(b,F')}{F\ a}{b \cdot {\tt{run}}\ F'\ s}
    \end{array}
\end{equation}




\section{coiterators}
The type of concrete streams of type \(a\) is defined as 
\(\co\ a = \Co\ (s \rightarrow a \times s)\ s\)
where \(s\) is the type of states used to generate the actual stream
and init is the initial state.
Given a value \(Co~h~s\), \(s\) is the initial state of the concrete stream and \(h\) is 
its step function. A usual stream is obtained through the \({\tt{run}}\) function defined below
\begin{equation*}    
    {\tt{run}}\ (\Co\ h\ s) = \letin{(a,s')}{h\ a}{{\tt{run}}\ (\Co\ h\ s')}
\end{equation*}
As an example consider the stream of natural numbers defined by 
\(\Co\ (\lambda n.(n,S\ n))\ 0\) of type \(co~nat\)
We have a family of functor indexed by type of states 
    $$F_S~X = (S \rightarrow X \times S) \times S$$

Operators define functions over concrete stream 

The {\tt{arr}} primitive apply a function to all elements of a streams 
$$
\begin{array}{lcl}
    head~(arr~f~s) &=& f~(head~s)\\
    tail~(arr~f~s) &=& arr~f~(tail~ s)
\end{array}
$$

This behavior can be obtained 
Given a concrete stream $c$, the $arr~f~c$ generates a new concrete
stream.
Arrow is a simple mapping of a function to all elements of a stream.
It is the mapping function for the functor $F$.

At each step, it uses $c$ to generate a value and a state 
and returns the vale (f a) and the new state.



\section{Yampa with coiterators}
Combinators have type \(\YSF{a}{b} = SF\ (\raw~a~s_1 \rightarrow \raw~b~(s_1 \times s_2))\) where,
intuitively, $s_2$ is the additional state introduced by the combinator.
When the combinator doest not need to increase the state size, i.e. when it does not contain
loops, \(s_2\) is simply \(\unittype\). This may seem 

\begin{equation*}
\begin{array}{lcl}
    \YArrow & : & (a \rightarrow b) \rightarrow (\raw\ a\ s) 
        \rightarrow (\raw\ a\ (s \times \unittype))\\
    \YArrow~f & = & \lambda~(\Co\ h\ s).
        \Co (\lambda (s,\unitval). \letin{(a,s')}{h\ s}{(f~a, (s',\unitval))})\ (s,\unitval)\\\\
    \YFirst & : & (\raw\ a\ s_1 \rightarrow \raw\ b\ (s_1 \times s_2)) \rightarrow 
                      (\raw\ (a \times c)\ s_1 \rightarrow \raw\ (b \times c)\ (s_1 \times s_2))\\
    \\\\
    \YCompose & : & (\raw\ a\ s_1 \rightarrow \raw\ b (s_1\times s_2)) \rightarrow \\
    && \qquad (\raw\ b\ (s_1 \times s_2) \rightarrow \raw\ c ((s_1\times s_2) \times s_3)) \rightarrow \\
    && \qquad\qquad (\raw\ a\ s_1 \rightarrow \raw\ c (s_1 \times (s_2 \times s_3)))\\
    \\\\
    \YLoop & : & (\raw\ (a \times x)\ s_1 \rightarrow \raw\ (b \times x)\ (s_1 \times s_2))
        \rightarrow \\
    && \qquad x \rightarrow \raw\ a\ s_1 \rightarrow \raw\ b\ s_1 \times (s_2 \times s_3) 
\end{array}
\end{equation*}

\begin{equation}
    \begin{array}{lcl}
        dummy &:& \raw\ \unittype\ \unittype\\
        dummy &=& \Co\ dup\ \unitval\\\\
        natural &:& \raw\ nat\ nat\\
        natural &=& \Co\ ((mapright\ S << dup))\ 0\\\\
    \end{array}
\end{equation}

\begin{equation}
    \begin{array}{lcl}
        identity &:& \co\ a \ s \rightarrow \co\ a \ (s \times \unittype)\\
        identity &=& \YArrow (\lambda x.x)\\\\ 
        plusleft &:& \co\ int\ s \rightarrow \co\ (int * int)\ (\unittype \times (\unittype \times \unittype))\\
        plusleft & = & \YArrow\ dup \YCompose \YFirst\ (\YArrow\ S)\\\\
        counter & : & \co\ \unittype\ \unittype \rightarrow \co\ {\tt{nat}}\ (\unittype \times (\unittype \times \unittype))\\
        counter & = & \YLoop\ (\YArrow ({\mathit{mapright}}\ S) \circ {\mathit{dup}} \circ {\mathit{snd}})\ 0\\\\
        pre &:& a \rightarrow \co\ a\ s \rightarrow \co\ a\ (s \times (\unittype \times a))\\
        pre &=& \YLoop (\YArrow {\mathit{swap}})  
    \end{array}
\end{equation}

% let plus_left = arr dup >>> first (arr (( + ) 1))

% let squares = arr dup >>> (arr (uncurry ( * )))

% (* STREAM FUNCTIONS WITH LOOPS *)

% let counter = loop (arr (mapright (( + ) 1) << dup << snd)) 0
  
% let pre = loop (arr swap)

% let sum = loop (arr (dup << uncurry ( + ))) 0

% let loop : ((('a * 'x), 's1) co -> (('b * 'x), 's1 * 's2) co) -> 
%     'x -> ('a, 's1) co -> ('b, 's1 * ('s2 * 'x)) co = 

% let first : 'a  'b 'c 's1 's2. 
%   (('a, 's1) co -> ('b, 's1 *'s2) co) -> 
%     ('a * 'c, 's1) co -> ('b * 'c, 's1 * 's2) co =
%   fun f (Co (h, s))  ->
%     let Co (h1, (s1,s2)) = f (Co ((mapleft fst << h), s)) in
%       Co (
%           (fun (s1,s2) -> 
%             let (a, (s1',s2')) = h1 (s1,s2) 
%             and b = fst ((mapleft snd << h) s1) in 
%               (a,b), (s1',s2')
%           ), (s1,s2))

% let (>>>) : 
%   (('a,'s1) co -> ('b, 's1 * 's2) co) -> 
%     (('b,'s1 * 's2) co -> ('c, ('s1 * 's2) * 's3) co) -> 
%       (('a,'s1) co -> ('c, 's1 * ('s2 * 's3)) co) = 
%   fun f g c -> 
%     let Co (h3, s) = g (f c) in
%       Co ((mapright permutright) << h3 << permutleft, permutright s)
      

% (** The unary [loop] operator plugs a stream function to a register *)            

% let loop : ((('a * 'x), 's1) co -> (('b * 'x), 's1 * 's2) co) -> 
%     'x -> ('a, 's1) co -> ('b, 's1 * ('s2 * 'x)) co = 
%   fun f x0 -> 
%     fun (Co (h, s1) : ('a, 's1) co) -> 
%       let g = fun x -> aux2 (f (aux1 (Co (h, s1)) x)) x in 
%     Co ((fun (s1, (s2,x)) -> 
%       let Co (h',_) = g x in h' (s1,(s2,x))), 
%       let Co (_,s) = g x0 in s)


The arrow combinator adds no additional state, thus the return type \(\raw\ a\ (s \times \unittype)\).

Let \({\tt{Type}}\) be the bicartesian closed category of types. 
We define the category $C$ as the category whose objects are
the \(\Sigma_{s:{\tt{Type}}}\ \raw\ a\ s\) for every \(a : {\tt{Type}}\)
and morphisms from object \(a\) to object \(b\) belongs to
\begin{equation*}
    \forall s_1,s_2:{\tt{Type}}. \raw\ a\ s_1 \rightarrow \raw\ b\ (s_1 \times s_2)
\end{equation*}

\begin{lemma}
    For all \(a\ b\ c : {\tt{Type}}\) and for all 
    \begin{itemize}
    \item $F : \forall s_1\ s_2. \raw\ a\ s_1 \rightarrow \raw\ b\ (s_1 \times s_2)$,
    \item $G : \forall s_1\ s_2.\raw\ b\ s_1 \rightarrow \raw\ c\ (s_1 \times s_2)$,
    \item $H : \forall s_1\ s_2.\raw\ c\ s_1 \rightarrow \raw\ d\ (s_1 \times s_2)$,
    \end{itemize}
    The following equalities hold.
    \begin{itemize}
    \item \(identity_a \YCompose F = F\) 
    \item \(F \YCompose identity_b = F \) 
    \item \((F \YCompose G) \YCompose H = F \YCompose (G \YCompose H) \) 
    \end{itemize}
\end{lemma}

\begin{itemize}
    \item all frp operators are polymorphic in the state of the coiterators passed as input 
    (which is a good thing, the computation should not depend on the representation)
    f l1 = f l2 si l1 et l2 représentent les memes suites de valeurs
\end{itemize}

\begin{itemize}
    \item sf are function that transform concrete streams 
    \item example where we augment the state
\end{itemize}

\section{Ocaml library}

\section{Coq formalization}

\begin{itemize}
    \item Program proof+ extraction
    \end{itemize}
\end{document}
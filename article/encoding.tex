Combinators have type \(\YSF{a}{b} = SF\ (\raw~a~s_1 \rightarrow \raw~b~(s_1 \times s_2))\) where,
intuitively, $s_2$ is the additional state introduced by the combinator.
When the combinator doest not need to increase the state size, i.e. when it does not contain
loops, \(s_2\) is simply \(\unittype\). This may seem 

\begin{equation*}
\begin{array}{lcl}
    \YArrow & : & (a \rightarrow b) \rightarrow (\raw\ a\ s) 
        \rightarrow (\raw\ a\ (s \times \unittype))\\
    \YArrow~f & = & \lambda~(\Co\ h\ s).
        \Co (\lambda (s,\unitval). \letin{(a,s')}{h\ s}{(f~a, (s',\unitval))})\ (s,\unitval)\\\\
    \YFirst & : & (\raw\ a\ s_1 \rightarrow \raw\ b\ (s_1 \times s_2)) \rightarrow 
                      (\raw\ (a \times c)\ s_1 \rightarrow \raw\ (b \times c)\ (s_1 \times s_2))\\
    \\\\
    \YCompose & : & (\raw\ a\ s_1 \rightarrow \raw\ b (s_1\times s_2)) \rightarrow \\
    && \qquad (\raw\ b\ (s_1 \times s_2) \rightarrow \raw\ c ((s_1\times s_2) \times s_3)) \rightarrow \\
    && \qquad\qquad (\raw\ a\ s_1 \rightarrow \raw\ c (s_1 \times (s_2 \times s_3)))\\
    \\\\
    \YLoop & : & (\raw\ (a \times x)\ s_1 \rightarrow \raw\ (b \times x)\ (s_1 \times s_2))
        \rightarrow \\
    && \qquad x \rightarrow \raw\ a\ s_1 \rightarrow \raw\ b\ s_1 \times (s_2 \times s_3) 
\end{array}
\end{equation*}

\begin{equation}
    \begin{array}{lcl}
        dummy &:& \raw\ \unittype\ \unittype\\
        dummy &=& \Co\ dup\ \unitval\\\\
        natural &:& \raw\ nat\ nat\\
        natural &=& \Co\ ((mapright\ S << dup))\ 0\\\\
    \end{array}
\end{equation}

\begin{equation}
    \begin{array}{lcl}
        identity &:& \co\ a \ s \rightarrow \co\ a \ (s \times \unittype)\\
        identity &=& \YArrow (\lambda x.x)\\\\ 
        plusleft &:& \co\ int\ s \rightarrow \co\ (int * int)\ (\unittype \times (\unittype \times \unittype))\\
        plusleft & = & \YArrow\ dup \YCompose \YFirst\ (\YArrow\ S)\\\\
        counter & : & \co\ \unittype\ \unittype \rightarrow \co\ {\tt{nat}}\ (\unittype \times (\unittype \times \unittype))\\
        counter & = & \YLoop\ (\YArrow ({\mathit{mapright}}\ S) \circ {\mathit{dup}} \circ {\mathit{snd}})\ 0\\\\
        pre &:& a \rightarrow \co\ a\ s \rightarrow \co\ a\ (s \times (\unittype \times a))\\
        pre &=& \YLoop (\YArrow {\mathit{swap}})  
    \end{array}
\end{equation}

% let plus_left = arr dup >>> first (arr (( + ) 1))

% let squares = arr dup >>> (arr (uncurry ( * )))

% (* STREAM FUNCTIONS WITH LOOPS *)

% let counter = loop (arr (mapright (( + ) 1) << dup << snd)) 0
  
% let pre = loop (arr swap)

% let sum = loop (arr (dup << uncurry ( + ))) 0

% let loop : ((('a * 'x), 's1) co -> (('b * 'x), 's1 * 's2) co) -> 
%     'x -> ('a, 's1) co -> ('b, 's1 * ('s2 * 'x)) co = 

% let first : 'a  'b 'c 's1 's2. 
%   (('a, 's1) co -> ('b, 's1 *'s2) co) -> 
%     ('a * 'c, 's1) co -> ('b * 'c, 's1 * 's2) co =
%   fun f (Co (h, s))  ->
%     let Co (h1, (s1,s2)) = f (Co ((mapleft fst << h), s)) in
%       Co (
%           (fun (s1,s2) -> 
%             let (a, (s1',s2')) = h1 (s1,s2) 
%             and b = fst ((mapleft snd << h) s1) in 
%               (a,b), (s1',s2')
%           ), (s1,s2))

% let (>>>) : 
%   (('a,'s1) co -> ('b, 's1 * 's2) co) -> 
%     (('b,'s1 * 's2) co -> ('c, ('s1 * 's2) * 's3) co) -> 
%       (('a,'s1) co -> ('c, 's1 * ('s2 * 's3)) co) = 
%   fun f g c -> 
%     let Co (h3, s) = g (f c) in
%       Co ((mapright permutright) << h3 << permutleft, permutright s)
      

% (** The unary [loop] operator plugs a stream function to a register *)            

% let loop : ((('a * 'x), 's1) co -> (('b * 'x), 's1 * 's2) co) -> 
%     'x -> ('a, 's1) co -> ('b, 's1 * ('s2 * 'x)) co = 
%   fun f x0 -> 
%     fun (Co (h, s1) : ('a, 's1) co) -> 
%       let g = fun x -> aux2 (f (aux1 (Co (h, s1)) x)) x in 
%     Co ((fun (s1, (s2,x)) -> 
%       let Co (h',_) = g x in h' (s1,(s2,x))), 
%       let Co (_,s) = g x0 in s)


The arrow combinator adds no additional state, thus the return type \(\raw\ a\ (s \times \unittype)\).

Let \({\tt{Type}}\) be the bicartesian closed category of types. 
We define the category $C$ as the category whose objects are
the \(\Sigma_{s:{\tt{Type}}}\ \raw\ a\ s\) for every \(a : {\tt{Type}}\)
and morphisms from object \(a\) to object \(b\) belongs to
\begin{equation*}
    \forall s_1,s_2:{\tt{Type}}. \raw\ a\ s_1 \rightarrow \raw\ b\ (s_1 \times s_2)
\end{equation*}

\begin{lemma}
    For all \(a\ b\ c : {\tt{Type}}\) and for all 
    \begin{itemize}
    \item $F : \forall s_1\ s_2. \raw\ a\ s_1 \rightarrow \raw\ b\ (s_1 \times s_2)$,
    \item $G : \forall s_1\ s_2.\raw\ b\ s_1 \rightarrow \raw\ c\ (s_1 \times s_2)$,
    \item $H : \forall s_1\ s_2.\raw\ c\ s_1 \rightarrow \raw\ d\ (s_1 \times s_2)$,
    \end{itemize}
    The following equalities hold.
    \begin{itemize}
    \item \(identity_a \YCompose F = F\) 
    \item \(F \YCompose identity_b = F \) 
    \item \((F \YCompose G) \YCompose H = F \YCompose (G \YCompose H) \) 
    \end{itemize}
\end{lemma}

\begin{itemize}
    \item all frp operators are polymorphic in the state of the coiterators passed as input 
    (which is a good thing, the computation should not depend on the representation)
    f l1 = f l2 si l1 et l2 représentent les memes suites de valeurs
\end{itemize}

\begin{itemize}
    \item sf are function that transform concrete streams 
    \item example where we augment the state
\end{itemize}